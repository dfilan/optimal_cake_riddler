\documentclass[12pt]{article}
\usepackage{geometry}                
\geometry{a4paper}                   
\usepackage{graphicx}
\usepackage{amssymb}
\usepackage{epstopdf}
\usepackage{amsmath}
\usepackage{amsthm}
\usepackage{mathtools}
\usepackage{pdfpages}


\title{Baking the Optimal Cake}
\author{Daniel Filan}

\begin{document}
\maketitle

\section{Main problem}
\label{sec:main-problem}

\subsection{TL;DR}
\label{sec:tl-dr}

The height of the lowest layer is $205/1263 \approx 0.162$ times the height of the cone, the height of the middle layer is $230/1263 \approx 0.182$ times the height of the cone, and the height of the top layer is $276/1263 \approx 0.219$ times the height of the cone. The volume of cake is $1,119,364/1,595,169 \approx 0.702$ times the volume of the cone.

\subsection{Proof}
\label{sec:proof}

Let the height of the cone be $h$ and the radius of its base be $r$ ($r$ will end up being irrelevant to the solution, but we include it for the intermediate analysis). Suppose we have three layers of cake inside, and each layer is a cylinder. Denote the height of the bottom layer as $h_1$ and its radius as $r_1$, the height of the middle layer as $h_2$ and its radius as $r_2$, and the height of the top layer as $h_3$ and its radius as $r_3$. Since we are trying to maximise cake volume, we may assume that the radius of each layer is as large as possible while still fitting inside the cone. Therefore, the triangle between the tip of the cone, the middle of the base of the cone, and some point on the circumference of the base of the cone will be similar to the triangle between the tip, the middle of the top of the first layer, and a point on the circumference of the top of the first layer. This gives us
\begin{align*}
  \frac{r_1}{h-h_1} &= \frac{r}{h} \\
  \therefore r_1 &= (h-h_1) \frac{r}{h}
\end{align*}
and similarly
\begin{equation*}
  r_2 = (h - h_1 - h_2) \frac{r}{h},\quad r_3 = (h - h_1 - h_2 - h_3) \frac{r}{h}
\end{equation*}

We can now compute the volume of cake $V$ as a function of the heights $h_1, h_2, h_3$ using the above expressions for the radii:
\begin{align*}
  V(h_1, h_2, h_3) &= \pi (r_1^2 h_1 + r_2^2 h_2 + r_3^2 h_3) \\
  &= \frac{\pi r^2}{h^2}(h_1 (h - h_1)^2 + h_2(h - h_1 - h_2)^2 + h_3(h - h_1 - h_2 - h^3)^2)
\end{align*}
Our task is to maximise this function under the constraints that $h_1 \geq 0$, $h_2 \geq 0$, $h_3 \geq 0$, and $h_1 + h_2 + h_3 \leq h$. We will do this by finding the value of the vector $(h_1, h_2, h_3)$ such that $\nabla V(h_1, h_2, h_3) = 0$. Since there will only be one such vector, the maximum will be either at that point, or on the boundary of the allowed region, in which case every point on the boundary will have greater volume than our solution. However, there are points on the boundary with volume 0 (namely $h_1 = h_2 = h_3 = 0$), and we can certainly make cakes with positive volume, so the solution we find will indeed be a maximum.

Setting the partial derivatives to 0,

\begin{align*}
  \frac{\partial V}{\partial h_3} &= \frac{\pi r^2}{h^2} \left( (h - h_1 - h_2 - h_3)^2 - 2h_3(h - h_1 - h_2 - h_3) \right) \\
  &= \frac{\pi r^2}{h^2}(h - h_1 - h_2 - h_3)(h - h_1 - h_2 - 3h_3)
\end{align*}
For this to vanish, either $h_1 + h_2 + h_3 = h$ or $h - h_1 - h_2 - 3h_3 = 0$. But if $h_1 + h_2 + h_3 = h$, that means that one cake is a cylinder with no volume reaching to the very peak of the cone. We could then increase the volume my making that cylinder shorter, so this would not be our optimum. Therefore,
\begin{equation}
  \label{eq:2}
  h - h_1 - h_2 - 3h_3 = 0
\end{equation}

Next,
\begin{align*}
  \frac{\partial V}{\partial h_2} &= \frac{\pi r^2}{h^2} \left( (h - h_1 - h_2)^2 - 2h_2 (h - h_1 - h_2) - 2h_3(h - h_1 - h_2 - h_3) \right) \\
                                  &= \frac{\pi r^2}{h^2} \left( (h - h_1 - h_2)(h - h_1 - 3h_2) - 2h_3 (h - h_1 -h_2 - h_3) \right) \\
  &= \frac{\pi r^2}{h^2} \left( (h - h_1 - h_2)(h - h_1 - 3h_2) - 4h_3^2 \right) \text{, from (\ref{eq:2})}\\
\end{align*}
Therefore,
\begin{equation}
  \label{eq:3}
  (h-h_1 - h_2)(h-h_1-3h_2) = 4h_3^2
\end{equation}

Finally,
\begin{align*}
  \frac{\partial V}{\partial h_1} &= \frac{\pi r^2}{h^2}\left( (h - h_1)^2 - 2h_1(h-h_1) - 2h_2(h - h_1 - h_2) - 2h_3(h - h_1 - h_2 - h_3) \right)\\
                                  &= \frac{\pi r^2}{h^2} \left( (h-h_1)(h-3h_1) - 2h_2(h-h_1 - h_2) - 4h_3^2 \right)\text{, from (\ref{eq:2})} \\
                                  &= \frac{\pi r^2}{h^2} \left( (h-h_1)(h-3h_1) - 2h_2(h-h_1 - h_2) - (h-h_1-h_2)(h-h_1-3h_2) \right), \\
                                  &\quad \text{from (\ref{eq:3})}\\
  &= \frac{\pi r^2}{h^2}\left( (h-h_1)(h-3h_1) - (h-h_1 - h_2)^2 \right)
\end{align*}
Therefore,
\begin{equation}
  \label{eq:4}
  (h-h_1)(h-3h_1) = (h-h_1 - h_2)^2
\end{equation}

Now, we simply solve (\ref{eq:2}), (\ref{eq:3}), and (\ref{eq:4}) to get the required solution:
\begin{align*}
  h_3 &= \frac{1}{3}(1-h_1-h_2) \\
  h - h_1 - 3h_2 &= \frac{4}{9}(h-h_1-h_2) \\
  h_2 &= \frac{5}{23}(h-h_1)\\
  h - 3h_1 &= \frac{18^2}{23^2} (h-h_1) \\
  h_1 &= \frac{205}{1263}h \\
  h_2 &= \frac{230}{1263}h \\
  h_3 &= \frac{276}{1263}h
\end{align*}

We can substitute these in to find the volume, $\pi r^2 h \times 1,119,364/4,785,507$, and remembering that the volume of the cone is $\pi r^2 h / 3$ gives us the fraction of the volume of the cone that the cake occupies.

\end{document}
%%% Local Variables:
%%% mode: latex
%%% TeX-master: t
%%% End:
